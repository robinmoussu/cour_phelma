\documentclass[]{article}
\usepackage[T1]{fontenc}
\usepackage{lmodern}
\usepackage{amssymb,amsmath}
\usepackage{ifxetex,ifluatex}
\usepackage{fixltx2e} % provides \textsubscript
% use upquote if available, for straight quotes in verbatim environments
\IfFileExists{upquote.sty}{\usepackage{upquote}}{}
\ifnum 0\ifxetex 1\fi\ifluatex 1\fi=0 % if pdftex
  \usepackage[utf8]{inputenc}
\else % if luatex or xelatex
  \usepackage{fontspec}
  \ifxetex
    \usepackage{xltxtra,xunicode}
  \fi
  \defaultfontfeatures{Mapping=tex-text,Scale=MatchLowercase}
  \newcommand{\euro}{€}
\fi
% use microtype if available
\IfFileExists{microtype.sty}{\usepackage{microtype}}{}
\ifxetex
  \usepackage[setpagesize=false, % page size defined by xetex
              unicode=false, % unicode breaks when used with xetex
              xetex]{hyperref}
\else
  \usepackage[unicode=true]{hyperref}
\fi
\hypersetup{breaklinks=true,
            bookmarks=true,
            pdfauthor={},
            pdftitle={},
            colorlinks=true,
            urlcolor=blue,
            linkcolor=magenta,
            pdfborder={0 0 0}}
\urlstyle{same}  % don't use monospace font for urls
\setlength{\parindent}{0pt}
\setlength{\parskip}{6pt plus 2pt minus 1pt}
\setlength{\emergencystretch}{3em}  % prevent overfull lines
\setcounter{secnumdepth}{0}

\author{}
\date{}

\begin{document}

\newcommand \differential[2] {
    \frac{\mathrm{d} #1}{\mathrm{d}#2}
}
\newcommand \dd[1] {
    \mathrm{#1}
}

\section{Probabilités et statistiques}

\subsection{INTRODUCTION}

au 17ème siècle PASCAL et DE FERMAT ont basé leur travaux sur la Théorie
des jeux. =\textgreater{} À donné naissance à toutes la théorie
combinatoire.

Au 20ème siècle, KOLMOGOROV à formalisé les statistiques pour arriver au
formalisme moderne. Ses travaux sont basés sur la théorie de la mesure).

\subsubsection{Notion intuitive de probabilité}

\begin{itemize}
\item
  Sur 1D6, la probabilité d'obtinir un 5 est de $\frac{1}{6}$ uniquement
  si le dé est équilibré.
\item
  Soit N lancés de 1D6 équilibré. La probabilités d'obtenir un 5 n'est
  valable que pour N grand.
\end{itemize}

\subsection{MESURE ET PROBABILITÉS}

\subsubsection{Vocabulaire}

Une Épreuve aléatoire $\Omega =$ \{résultats possibles\}\\$\omega_p =$
résultats élémentaires

\subsubsection{exemple du dé}

$\omega_1 = 1, \omega_2 =2$\\$\Omega = \{ 1, 2, 3, 4, 5, 6 \}$\\$A \subset \Omega$
évènements = résultats possibles

$A$ = famille de $A_i$ vérifiant 3 règles :

\begin{itemize}
\itemsep1pt\parskip0pt\parsep0pt
\item
  $\Omega \in A$
\item
  $\forall A_i \in A, C_\Omega ^{A_i} \in A$
\item
  $\forall A_i \in A, i \in A_i \in A$
\end{itemize}

L'ensemble forme une tribut $A$

\subsubsection{Définition}

On associe à chaque $A_i \in A$ un nombre $\equiv$ probabilité

\begin{displaymath}
\begin{array}{r l}
p: & A \rightarrow [O,1] \\
   & A_i \rightarrow p(Ax) = \text{probabilité de réaliser} A_i \text{sur une épreuve.} \\
\end{array}
\end{displaymath}

P vérifie 3 règles:

\[A_i \exists A, p(\emptyset) = 0 \leq p(A_i) \leq 1 = p(\Omega)\]
\[A_i \exists A, p(C_\Omega A_i) = 1 - p(A_i)\]
\[A_i \exists A, i \in I p(U(i\in I) A_i)  = \sum_{i\in I} p(A_i)\]

ssi les $A_i$ sont 2 à 2 disjoints.

\subsubsection{exemple}

dé à 6 faces

\begin{displaymath}
\begin{array}{r l}
\Omega & = \{1,2,3,4,5,6\} \\
A_1 & = \{\Omega, \emptyset\} \\
A_2 & = \{\Omega, \emptyset, \{1,3,5\},\{2,4,6\}\} \\
A_3 & = \{\Omega, \emptyset, \{1\}, \{2\}, \{2,3,4,5,6\}, \{1,3,4,5,6\}, \{1,2\}, \{1,4,5,6\}\} \\
(\Omega, A, p) & \equiv \text{espace probabilisé}
\end{array}
\end{displaymath}

\begin{tabular}{r l}
$A_i$ et $A_j$ incompatible & $p(A_i \cap A_j) = 0$ \\
$A_i$ et $A_j$ indépendants & $p(A_i \cap A_j) = p(A_i)\times p(A_j)$ \\
Probabilité inconditionnelle & $p(A_i \mid A_j) = \frac{p(A_i \cap A_j)}{p(A_j)}$ \\
\end{tabular}

\begin{tabular}{r l}
proba & $A_i$ conditionnelement à $A_j$ \\
      & $A_i$ sachant $A_j$
\end{tabular}

\subsubsection{Loi des probabilités totales}

partition $(A_i)$ de
$\omega \in A \rightarrow \cup_i A_i = \Omega, \forall i \neq j, A_i \cap A_j = \emptyset$

$\forall B \in A, p(B) = \sum\limits_{i\in I} p(B\mid A_i) p(A_i)$

\subsubsection{Formules de Bayes}

Partition $(A_i)$ de $\Omega$ : $\forall B \in A$

\begin{displaymath}
\begin{array}{r l}
p(A_k \mid B) &= \frac{p(B \mid A_k) p(A_k)}{\sum_i p(B \mid A_i ) p(A_i)} \\
              &= \frac{p(A_k \cap B)}{p(B)} \\
\end{array}
\end{displaymath}

$B = \cup_i (B \cap A_i) \equiv$ partition de
B.\\$\rightarrow p(B) = \sum_i p(B \cap A_i) = \sum_i p(A \mid A_i) p(A_i)$

\subsubsection{Propriété de la mesure de proba}

$\forall A_i$ et $A_j$ disjoints
$p(A_i \cap A_j) = p(A_i) + p(A_j)$\\Dans le cas général
:\\$p(A_i \cap A_j) = p(A_i) + p(A_j) -p(A_i \cap A_j)$

\textbf{Voir schéma 1}

\subsubsection{Exercice}

Je cherche mon cours, que j'ai rangé dans mon bureau avec une
probabilité p.~Le bureau possède 3 tiroirs semblables.

\begin{itemize}
\itemsep1pt\parskip0pt\parsep0pt
\item
  J'ouvre le 1er tiroir \ldots{} en vain
\item
  j'ouvre le 2ème tiroir \ldots{} en vain
\end{itemize}

Quelle est la probabilité de le trouver dans le 3 ème tiroir ?

Soit $A$ le cour dans le 3ème tiroir.\\Soit $B$ le cour ni dans le 1er,
ni dans le 2ème.

\begin{displaymath}
\begin{array}{r l}
p(A \mid B) &= \frac{p(A \cap B)}{p(B)} \\
            &= \frac{p(A)}{p(B)} \text{ car A est compris dans B}
\end{array}
\end{displaymath}

Au départ : $\Omega = \{\omega_1, \omega_2, \omega_3, \omega_0\}$\\avec
$\omega_0 =$ pas dans le bureau.\\$\omega_i =$ dans le tiroir
$i (i = 1,2,3)$

La tribu utile sur cet exemple :

\begin{displaymath}
\begin{array}{r l}
A_1 &= \{\Omega, \emptyset, \{\omega_1\}, \{\omega_2\}, \{\omega_1,\omega_2\}, \{\omega_2,\omega_3,\omega_0\},…\} \\
A_2 &= \{\Omega, \emptyset, \{\omega_1,\omega_2\}, \{\omega_3,\omega_0\},\{\omega_0\}, \{\omega_3\},\{\omega_1,\omega_2,\omega_3\},\{\omega_0,\omega_1,\omega_2\}\} \\
\end{array}
\end{displaymath}

Pour $A_2$, les probabilités sont respectivement de
$\{1,0,\frac{2\cdot}{3},1-p+\frac{p}{3},1-p,\frac{p}{3},p,1-p+\frac{2\cdot p}{3}\}$

\begin{displaymath}
A_2' = \{\{\omega_3,\omega_0\},\emptyset,\{\omega_3\},\{\omega_0\}\}
\end{displaymath}

\subsection{VARIABLE ALEATOIRE RÉELLE CONTINUE}

Soit une variable aléatoir x à valeurs sur $\mathbb{R}$

\[ \Omega = \mathbb{R} \] \[ A_i \subset \Omega \]
\[ A = \{A_i\} = tribu \]

Les $A_i$ de base sont $]-\infty ; x ]$ $\hookrightarrow$ tribu
Borclienne $\equiv \mathcal{B}_{\mathbb{R}}$

\subsubsection{Fonction de répartition de la variable aléatoire x}

\begin{displaymath}
\begin{array}{rl}
F_x : &\mathbb{R} \rightarrow [0,1] \\
&x \in \mathbb{R} \rightarrow \text{proba} (x \in ]-\infty, x]) \\
\end{array}
\end{displaymath}

Il existe 3 types de variables aleatoires.

\begin{tabular}{l l}
- continue : &$F_x$ est continue. **cf fig 2** \\
- discrete : &$F_x$ est continue par morceaux. **cf fig 3** \\
- mixte : &(ex: temps d'attente à un feu) **cf fig 3** \\
\end{tabular}

propritétés de $F_x$ : \[\forall x \in \mathbb{R}\]
\[F_x (-\infty) \leq F_x (x) \leq 1 = F_x (+\infty)\]
\[F_x \text{ est croissante}\] \[\forall a, b \in \mathbb{R}, a \leq b\]
\[F_x(b) - F_x()\] \[]-\infty, b] \  ]-\infty, a]\]

\[F_x(x) = \int\limits_{-\infty}^{\infty} f_x (u) \mathrm{d}\]

fct répartition
\[f_x (x) = \differential{F_x}{x} \text{ (au sens des distributions)}\]

Densité de probabilité $f_x = \differential{F_x}{x} \leq 0$i.\\$f_x$ est
sommable :
$\int_{\mathbb{R}} f_x (x) \mathrm{x} = 1 \rightarrow \hat{f_x}$ existe
(le chapeau correspond à la transformée de Fourier).

\begin{tabular} {ll}
La probabilité de $(x \in ]a,b)$ &= $F_x(b) - F_x(a)$ \\
&= $\int\limits_a^b F_x(x) \mathrm{d}x$ \\
\end{tabular}

\subsubsection{exemple}

Loi uniforme sur {[}a,b{]}, avec b \textgreater{} a.

$f_x$ est constante sur {[}a,b{]}, et nulle ailleurs

\textbf{cf ex 5}

\subsubsection{Loi de Gauss}

C'est la même chose que la loi normale
$\mathrm{N} (m,\sigma^2). $$F_x(x) = \frac{1}{\sigma \sqrt{2\pi}} \exp{-\frac{x - m)^2}{2\sigma^2}}$\$

\textbf{cf fig 6}

NB : $F_x(x) \rightarrow$ table de loi

\subsubsection{Loi de Cauchy (a\textgreater{}0)}

\[f_x(x) = \frac{1}{\pi} \frac{a}{a^2 + x^2}\]
\[F_x(x) = \frac{1}{2} + \frac{1}{\pi}\arctan{\frac{x}{a}}\]

\subsubsection{Espérance mathématique d'une variable aléatoire x de
densité de probabilité $f_x$}

\boxed{E\{x\} = \int\limits_{\mathbb{R}} x f_x(x) \mathrm{d}}

NB : peut ne pas exister.

exemple : Cauchy $E(x)= +\infty$
\[E\{g(x)\} = \int\limits_{\mathbb{R} g(x)f_x(x) \mathrm{d}}\]

La fonction g doit être mesurable $\sim$ continue par morceaux, en
particulier : $g(x) = x^n$
\[E(x^n) = \int_{\mathbb{R}} x^n f_x(x) \mathrm{d}x\] On appelle cette
fonction un moments d'ordre n.

\begin{tabular}{l l}
n = 1 & $\rightarrow \text{moyenne de } x$ \\
n = 2 & $\rightarrow \text{dispertion de } x$ \\
\end{tabular}

Variance de x : $E\{[x-E(x)]^2\} = \text{var(x)}$ (NB: la partie entre
crochet est une variable aléatoire x contrée).

\begin{tabular}{r l}
$\text{var}(x)$ &= $E(x^2) -E^2(x)$ \\
&= $E\{x^2 - 2xE(x)+E^2(x)\}$ \\
&= $E (x^2) - 2E(x)E(x) + E^2(x)$ \\
&= $E (x^2) - E^2(x)$ \\
\end{tabular}

NB : $E$ est un opérateur binaire.

\begin{tabular}{l l}
E(x) &= $\int_\mathbb{R} x F_x(x) \mathrm{d}x$ \\
E(ax + b) &= $\int_\mathbb{R} (ax+b)f_x(x)\mathrm{d}x$ \\
&= $a\int_\mathbb{R} x f_x(x) \mathrm{d}x +b\int_\mathbb{R} x f_x(x) \mathrm{d}x$ \\
\end{tabular}

\subsubsection{Inégalité de Bienaginé - Chebychev}

Variable aléatoire x de densité de probabilité $f_x$. La probabilité
$(\lvert x - E(x)\rvert > a) \leq \frac{\sigma^2}{a^2}$

\begin{tabular}{l l}
$\sigma^2$ & $\equiv \text{var}(x)$ \\
$\sigma$ & $\equiv \sqrt{\text{var}(x)}$ \\
\end{tabular}

$\sigma$ est l'écart type.

\textbf{Cf démo dans le poly}

\subsubsection{Changement de variable}

v.a.x de la loi connue $f_x$.

\begin{itemize}
\itemsep1pt\parskip0pt\parsep0pt
\item
  Loi bijection croissante :
\end{itemize}

\[y = h(x)\] \[h \equiv \text{continue par morceaux}\]

\textbf{cf fig 7}

\begin{tabular}{r l}
$F_y(y)$ &= $p(y \leq y)$ \\
&= $p(x \leq h^{-1}(y))$ \\
&= $F_x(h^{-1}(y))$ \\
$F_y(y)$ &= $\differential{}{y} F_y(y)$ \\
&= $\differential{}{x} F_x(h^{-1}(y)) \differential{}{y}h^{-1}(y)$ \\
&= $f_x(h^{-1}(y))\differential{x}{y}$ \\
\end{tabular}

\subsection{DISCRETE}

\subsection{VECTORIELLE}

\subsection{REGRESSION}

\subsection{THEOREMES AUX LIMITES}

\subsection{ESTIMATION PARAMÈTREQUE}

\subsection{TESTS D'HYPETHÉSES}

\subsection{Références}

BASS : Éléments de calcul des probabilités MASSON VENTETSEL : Théorie
des probabilités MIR RÉNYI : Calcul des probabilités MASSON (I.GABAY)
JAFFARD : Méthodes de la statistique MASSON Série SCHAUM chamilo

\end{document}
